\chapter{Normalizing Flows}
\section{Introduction}
In the recent years Normalizing Flow (NF) has become a popular way to estimate
a target distribution by transforming a random variable from a simple distribution 
such as Gaussian or Uniform.
Normalizing flow is a flexible and often computationally cheap way
to estimate distributions, which allow for easy sampling and also evaluation of
likelihood. The core of NF are invertible functions \(f_i: \mathbb{R}^D \rightarrow 
\mathbb{R}^D\), where \(D\) is the dimension of the distribution. The functions are often 
referred to as transformations. 
To sample with NF, one starts with a sample \(z^0\) from a well 
known distribution such as Uniform or Gaussian, and then apply sequentially 
\begin{align}
z^{(i)} = f_i(z^{(i-1)}), i=1,2,3,...,M.
\end{align}
\(M\) can be one, but are often greater, i.e chain of transformations.\\
