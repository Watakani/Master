    %! TEX root = ../main.tex
\section{Universality}
An aspect of normalizing flows is to know whether or not we can approximate any
distribution by simple increasing some \(\phi\), where \(\phi\) can be either
one or several parameters determined before training. In some cases such as a \(\mathcal{NN}^{+}\)-flow,
we may talk about the size of the neural network.
With splines it may be the number of knots used. Another
apt parameter in the context of normalizing flows is the number of transformations. 
The important part is that we can guarantee convergence in
distribution given flow, by increasing some parameters \(\phi\). We also denote
\(\bm z_T^{(\phi)}\) as the result of a corresponding flow \(f\) using \(\phi\)-neurons/knots/transformation etc.
It is worth noting that even with this guarantee,
we may not do well with any given target distribution in practice, as we have finite data 
and computing power to estimate the parameters. This is quite analogous to neural networks in general,
as we know we can approximate any continuous function when the input is compact. However,
when working with images one introduces an inductive bias regarding locality
which results in a Convolutional Neural Network (CNN). Regardless, knowing 
that a given flow is universal w.r.t distribution tells us that the target space
induced by the flow is not restricitive. We follow \cite{naf} when formally defining universality.
\begin{definition} 
    Let \((\mathcal{A}, f)\) be a normalizing flow, \(p_{\bm x}\) be any target density in a class of densities \(\mathscr{P}\),
    and \(\phi\) be a set of parameters which by increasing also increases the size of the model.
    The flow \(f\) is an \emph{universal density approximator} (UDA) for \(\mathscr{P}\) if there exist
    a sequence of \(\bm z_T^{(\phi)}\), such that when \(\phi \rightarrow \infty\),
    \(\bm z_T^{(\phi)} \xrightarrow{d} \bm x\).
\end{definition}
A couple of remarks are in order. Firstly, This definition is quite analogous to how we speak about universal approximators with regards to neural networks. We have a certain
network architecture and class of functions, namely continuous functions. We then posit existence of a specific network which converges
to any function in the class uniformly, when increasing the model through increasing \(\phi\), where \(\phi\) can in this case be number
of layers or number of neurons. 
Secondly, we do recognize that the name \emph{universal density approximator} is a little bit misleading. If something converges in 
distribution does not imply convergence in density. This can be illustrated with a density
\begin{align*}
    p_{n}(\bm x) = 
    \begin{cases}
        1 - \cos(2\pi n x), & \text{if \(0 < x < 1\)} \\
        0,& \text{elsewhere} 
    \end{cases}
\end{align*}
This does clearly not converge to any density, but the CDF of \(p_n\) is equal to 
\(x - \frac{\sin(2\pi n x)}{2 \pi n}\) which converges to \(x\) when \(n \rightarrow \infty\). Hence,
it converges to the Uniform distribution between 0 and 1, while the density does not converge to 1. 
One might ask how well the density we achieve from the flow corresponds to the target
density, even in flows that are UDA. Through the way one train the flow, and its 
use of KL-divergence, makes it less probable that we converge in distribution while 
not being reasonably close density wise. It may, however, affect the ability of finding the parameters
such that we have convergence in distribution. We therefore ought to be cautious of the property UDA and 
what is conveys, while still appreciate the guarantee that we can in theory converge to any continuous target distribution.
Also worth noting that if one is able to show that the density of a flow can converge to any target density, then
the flow is an UDA. This follows from Scheffé's Theorem, which states that convergence in density a.e implies convergence 
in distribution (\cite{scheffe}). Regardless, we follow the literature and its definition of an UDA, while keeping in mind
the gap between convergence of density and in distribution.
\subsection{NAF}
As alluded to earlier, \(\mathcal{NN}^{+}\)-flows are UDA. To be more specific, a certain model in the class of flows 
referred to as NAF is an UDA for \(\mathscr{P}\), where \(\mathscr{P}\)
is the class of positive, continuous densities. To quickly summarize the result presented by Huang et.al: The flow given by
an transformation function
\begin{align}\label{eq:naf}
    f_{d}(\bm z_{0}) = \sigma^{-1}\left(\sum_{j=1}^n w_{d,j} \sigma\left(\frac{z_{0,d} - b_{d,j}}{\tau_{d,j}}\right)\right),
    & \text{ for all \(d \in \mathcal{D}\)}
\end{align}
where the parameters \((w_{d,j}, b_{d,j}, \tau_{d,j})^{n}_{j=1}\) is given by \(\mathcal{H}(\mathcal{S}_{ext}(t,d))\).
The flow-structure is an IAR-structure and \(\mathcal{H}\) is a deep neural network. 
The flow described is referred to as NAF-DSF.
It is only one transformation, and all its expressiveness rely on increasing number of 
neurons in the hidden layer, \(n\). With this flow, Huang et.al proved the following theorem.
\begin{theorem}[\cite{naf}]
    Let \(\bm z_0\) be a random vector in an open set \(\mathcal{U} \subset \R^D\).
    Let \(\bm x\) be a random vector in \(\R^D\) with its density being a member of 
    \begin{align*}
        \mathscr{P} = \{p_{\bm x}\mid p_{\bm x}\in C(\R^D;\R_*^{+})\}.
    \end{align*}
    Then there exist a transformation
    of \(\bm z_0\) to \(\bm z_1\) through NAF-DSF such that 
    \(\bm z_1 \xrightarrow{d} \bm x\). 
\end{theorem}
Hence, NAF-DSF is an UDA. However, as mentioned earlier, the inverse of NAF is only tractable. A 
natural question is then whether or not a flow with analytical inverse can be an UDA.

\subsection{Affine transformation}
When it comes to flows with affine transformation, is has for a long time been an open
question whether or not the flow is an UDA. This year Wehenkel and Louppe gave a 
counter-example in terms of a target density that cannot be approximated by affine flow. 
The goal of this section is to both refute the counter-example for a general \(D\), 
and also prove that affine flow is not an UDA for \(D = 1\). We start by refreshing what
an affine normalizing flow is.
\begin{align}
    z_{0,d} &\sim q_{\bm{z}_0}\\
    z_{t,d} = a_{t,d}&\cdot z_{t-1,d} + b_{t,d}\label{eq:affine_trans}
\end{align}
for all \(t\in \mathcal{T}\) and \(d\in \mathcal{D}\), and with 
\((a_{t,d}, b_{t,d}) = \mathcal{H}(\mathcal{S}_{ext}(t,d))\). With the the following constraints,
that all \(a\)'s are positive, and that the structure is with permutations, i.e all dimensions have 
more than one ancestor in the corresponding graph. If this is not the case, then there exist a \(d\) such that 
\(z_d\) is an affine transformation of constants \(a\)'s and \(b\)'s, which is detrimental to
the flows expressiveness, as we shall see. 

A paper emerged (\cite{wehenkel}), which claimed that affine flow cannot be an UDA, regardless of structure 
or number of transformations. The proof is by a counter-example, which state a target density where one component is 
independent, and with base density being a Gaussian with diagonal covariance matrix. That is, a target distribution
with \(\bm x \sim p_{\bm x}\) such that \(x_j \indep x_{-j}\), for at least one \(1 \leq j \leq D\). The first argument 
that if the flow is construced such that \(z_j\) only have one ancestor, it cannot be an UDA. This is correct, as then 
\(z_j\) becomes Gaussian. However, with a reasonable structure that makes \(z_j\) have more than itself as ancestor, it 
is certainly not Gaussian (not even conditionally). The claim when the component has more than one ancestor is that 
it either must hurt the bijectivity of the flow, or independence. When the expectation and variance for \(p_{\bm x}\) exist,
we can show that this is not the case. We then claim that this also holds for when expectation and variance does not exist.

The idea for the proof is to construct part of the flow, and show that such flow exist. Then we are assuming that we can 
always find an affine flow from any non-independent distribution to another. Then it follows that any independent components 
in either base distribution, target distribution or both is not a problem given that affine flow is an UDA for non-independent
components. We have to diverge from the base distribution of Gaussian. This is simply to ensure we are sampling from a 
compact set, i.e closed and bounded on \(\R^D\). However, starting for example with a standard multivariate Gaussian
can also be approximated very well with a set that is large, yet bounded. Hence, theoretically
compactness is crucial, as it allow us to use universal approximation w.r.t deep neural networks. Yet practically, it 
ought not be too much difference from a very large compact set and sampling from a standard Gaussian. 

We are constructing part of the flow, which is the two first and two last parts of the flow. 
We are working with a target distribution
where each component is independt of the others and equally with a base distribution. Other versions with different causality
follows directly from this case. The flow we are working with start with the first transformation
\begin{align*}
    z_{1,d} = a(z_{0,1:d-1})&\,z_{0,d} + b(z_{0,1:d-1}) = z_{0,d} - \mu_{z_{0,d}}
\end{align*}
where \(\mu_{z_{0,d}}\) is the mean of the base density for the \(d\)th dimension. It then follows with 
\begin{align*}
    z_{2,d} = a(z_{1,1:d-1})&\,z_{1,d} + b(z_{1,1:d-1}) = z_{1,d} + \sum_{j=1}^{d-1} z_{1,j}.
\end{align*}
The final transformations are
\begin{align*}
    z_{T-1,d} = a(z_{T-1,1:d-1})&\,z_{T-2,d} + b(z_{T-1,1:d-1}) = z_{T-2,d} - \sum_{j=1}^{d-1} z_{T-1,j},
\end{align*} 
and finally
\begin{align*}
    z_{T,d} = a(z_{T,1:d-1})&\,z_{T,d} + b(z_{T,1:d-1}) = z_{T,d} + \mu_{z_{T,d}}.
\end{align*}
Notice that the structures for the first two steps are inverse autoregressive, while the latter two uses 
autoregressive structures. However, the last transformations can be written with an inverse autoregressive structure
by simply doing \(2*D\) transformations instead of 2, where for each transformation \(t\), one variable \(d\) is transformed by
as defined above, and the other dimensions are simply transformed by the identity function. Preferring fewer transformation
for readability, we stick with the transformation defined above, but noting that we can have an inverse autoregressive
structure. Additionally, we are going to talk about the density of the inverse transformation, that is \(q_{\bm z_{T-2}}^{-1}\). That is,
the density induced density of
\begin{align*}
    z_{T-2,d} =  (z_{T,d}-\mu_{z_{T,d}}) + \sum_{j=1}^{d-1} (z_{T,j}-\mu_{z_{T,j}}).
\end{align*}

We assume now that \(\bm z_D \sim p_{\bm x}\). Obviously in reality, we would not know what the
target distribution \(p_{\bm x}\) is, however we are here only interested to show that there
exist an affine flow from \(p_{\bm x}\) to some non-independent distribution. As the inverse is also an affine flow,
we can then use the existence of the inverse flow to construct a flow from non-independent to independent distribution. 
By then assuming that the affine flow is an UDA between non-independent distributions, we can show existence
of flow between independent distributions through our construction.

Let us then consider the density that the flow from transformation 0 to 1 induces, and the density 
induced by the inverse flow from \(T\) to \(T-1\). Since we are working with independence in both base and target density,
we have the induced densities
\begin{align*}
    q_{\bm z_1}(\bm z_1) = \prod_{d=1}^D q_{z_{0,d}}(z_{0,d})
\end{align*}
and
\begin{align*}
    q^{-1}_{\bm z_{T-2}}(\bm z_{T-2}) = \prod_{d=1}^D q_{z_{T,d}}(z_{T,d}) = \prod_{d=1}^D p_{x_{d}}(x_{d})
\end{align*}
where the last equality stems from the assumption above. Hence the density have not changed. However, we have 
added dependence.
\begin{lemma}\label{lemma:corr_densi}
    Let the expectation and variance exist for both the base and target density. 
    The two densities \(q_{\bm z_2}\) and \(q^{-1}_{\bm z_{T-2}}\) are both
    not independent.
\end{lemma}
\begin{proof}
    We only prove for one of the transformation, as the proof is identical. The
    expectation of any variable in \(\bm z_1\) is zero. That is,
    \begin{align*}
        E(z_{2,d}) = E\left(z_{1,d} + \sum_{j=1}^D z_{1,j}\right) 
        = E\left(z_{0,d} - \mu_{z_{0,d}} + \sum_{j=1}^D z_{0,j} - \mu_{z_{0,j}}\right) 
        = 0
    \end{align*}
    Which means that the covariance between any two variables \(z_{2,i}\) and \(z_{2,j}\), where w.l.g \(i<j\), is 
    \begin{align*}
        \text{Cov}(z_{2,i}, z_{2,j}) &= E\left(\left(z_{1,i} + \sum_{k=1}^{i-1} z_{1,j}\right) 
                                  \left(z_{1,j} + \sum_{l=1}^{j-1} z_{1,l}\right)\right)\\
                              &= E\left(\left(\sum_{k=1}^i \sum_{l=1}^i z_{1,k}\,z_{1,l}\right) + 
                                  \left(\sum_{k=1}^i \sum_{l=i+1}^j z_{1,k}\,z_{1,l}\right)\right).
    \end{align*}
    The expectation of the last sum is 0, as \(\bm z_1\) are independent (only subtracted mean from each dimension). 
    The first sum also end up in 0 when \(k \neq l\), hence
    \begin{align*}
        \text{Cov}(z_{2,i}, z_{2,j}) &= E\left(\sum_{k=1}^i z_{1,k}^2\right)\\
                                     &= E\left(\sum_{k=1}^i (z_{0,k}-\mu_{z_{0,k}})^2\right)
                                     = \sum_{k=1}^i \text{Var}(z_{0,k}) > 0.
    \end{align*}
    Which means that the induced distributions with density \(q_{\bm z_2}\) and \(q_{\bm z_{T-2}^{-1}}\) 
    are not independent.
\end{proof}
The next step before proving that such flows described above exist, given that we have affine flow being UDA
for non-independent distributions, is to make sure we have compactness after each flow. This is because we are
using a deep neural network to estimate the \(a\)'s and \(b\)'s for each transformation.
\begin{lemma}\label{lemma:compact}
    Let the set of values where the base density is greater than 0 be compact in \(\R^D\).
    Then the output is compact in \(\R^D\) for every transformation \(t=1,2,\cdots,T\) on an affine 
    flow with inverse autoregressive structure.
\end{lemma}
\begin{proof}
    A set in \(\R^D\) is compact if and only if it is bounded and closed. Assuming \(\bm z_{t-1} \in \mathcal{Z}_{t-1}\), 
    where \(\mathcal{Z}_{t-1}\) is a compact set in \(\R^D\). 
    Using, w.l.g, an inverse autoregressive structure where \(z_{t-1,d}\) is dependent on \(z_{t-1,1:d-1}\).
    
    To show preservation of compactness, we simply need to show that the transformation in \cref{eq:affine_trans} is 
    continuous. Transformation of the first dimension is clearly continuous, as the parameters \(a_{t,1}, b_{t,1}\) is
    is constant for all \(z_{t-1,1}\). Next, assuming the structure uses activation function ReLU, and final 
    activation function on \(a\) to make it positive is continuous, we have that the neural network with input \(z_{d-1,1:t-1}\)
    is continuous. As it simply is a composition of affine transformation with added ReLU (which is continuous), and an extra
    continuous activation function on \(a\). It therefore follows that the transformation of \(z_{t-1,d}\) is also continuous.
    Hence, the transformation of \(\bm z_{t-1}\) is continuous. Any continuous function from a compact subset of a metric space
    to another metric space implies the image of the function is also compact. 
\end{proof}
This allows us to always be able to find the transformations for \(t=1,2,T-2, T-1\), using inverse autoregressive structures
and allowing for the neural network associated with each structure to be as large as one needs. This is due to the fact
that we know input is compact and output is either constant, e.g subtracting mean, or sum of the input, which are both
continuous. Since the flows we wish to approximate are all continuous, hence we know we can find them using neural networks by \cref{thm:uni_nn}.
\begin{proposition}
    Let \(\mathcal{Z}_0 \in \R^D\) be compact. If affine flow with inverse autoregressive structure (with permutations) is an
    UDA when both base and target distribution are non-independent, then it is an UDA for all base and target densities with
    finite expectation and variance.
\end{proposition}
\begin{proof}
    As noted earlier, we are only proving it for when both base and target density are independent, since it follows from this that 
    it holds when either one is non-independent distributions. The flow goes as follows. Sample from \(\mathcal{Z}_0\), as the input
    is compact and both \(a\) and \(b\) in the first flow are constants, means that \cref{thm:uni_nn} can always approximate the
    first transformation
    \begin{align*}
        z_{1,d} = a(z_{0,1:d-1})&\,z_{0,d} + b(z_{0,1:d-1}) = z_{0,d} - \mu_{z_{0,d}}.
    \end{align*}
    From \cref{lemma:compact} and the fact that sum of input from a subset in \(\R^D\) is an continuous function, which means
    we can approximate transformation number two arbitraily well by \cref{thm:uni_nn}. That is, we can approximate 
    arbitraily well 
    \begin{align*}
        z_{2,d} = a(z_{1,1:d-1})&\,z_{1,d} + b(z_{1,1:d-1}) = z_{1,d} + \sum_{j=1}^{d-1} z_{1,j}.
    \end{align*}
    Through \cref{lemma:corr_densi} and assumption of UDA between non-independent distributions, we have that we 
    can find a flow such that the sample \(\bm z_{T-2}\) converges in distribution to the distribution with density \(q_{\bm z_{T-2}}^{-1}\).
    Through the same reasoning as with the first two transformations, assuming we rewrite the
    two last transformations such that it have an inverse autoregressive structure.
    As we know \(\mathcal{Z}_{T-2}\) is compact in \(\R^D\) due to
    \cref{lemma:corr_densi}, which means that we can arbitraily well approximate the transformations
    \begin{align*}
        z_{T-1,d} = a(z_{T-1,1:d-1})&\,z_{T-2,d} + b(z_{T-1,1:d-1}) = z_{T-2,d} - \sum_{j=1}^{d-1} z_{T-1,j}\\
        z_{T,d} = a(z_{T,1:d-1})&\,z_{T,d} + b(z_{T,1:d-1}) = z_{T,d} + \mu_{z_{T,d}}.
    \end{align*}
    By definition of the induced \(q_{\bm z_{T-2}}^{-1}\), i.e it is the inverse transformation is such that 
    \(\bm z_{T} \sim p_{\bm x}\). Which means that \(\bm z_0 \xrightarrow{d} \bm x\), which means that the affine flow
    is also an UDA for base and target distributions that are independent with expectation and variance existing.
\end{proof}
This means that, at least for the case when we have existence of expectation and variance, the counter-example given by \cite{wehenkel}
does not necessarily hold. We add necessarily, as we are currently working with base distribution defined on a compact set, which
we allow as a fair condition to put on base densities. Although we do acknowledge that the counter-example was based off independent
Gaussian, but rebuking our proposition due to the assumption of compactness is to refute all proofs of UDA in the literature. As all
proofs assume this, due to the limitations in \cref{thm:uni_nn}.  

\begin{conjecture}\label{conj:nonind}
    There always exist an affine flow with inverse autoregressive flow-structure, 
    such that any independent continuous distribution 
    can be transformed through the flow, where the induced distribution is non-independent.
\end{conjecture}
If this hold, then we strengthen our proposition such that if affine flow with inverse autoregressive structure is an UDA between
non-independent distributions, then it is also UDA for all distributions given compact \(\mathcal{Z}_0\). As we do not know
\cref{conj:nonind}, means that \cite{wehenkel} cannot be known as well.\\

However, we can easily prove that for \(D=1\), the affine flow cannot be an UDA. 
\begin{proposition}
    Let the target density be \(p_{\bm x}\) with \(D = 1\). Then an affine flow cannot be 
    an UDA, regardless of flow-structure.
\end{proposition}
\begin{proof}
    Due to the fact that all the parameters \(\{a_{t,1}, b_{t,1}\}^{T}_{t=1}\) in the flow are constants, as the set
    \(\mathcal{E}_{ext}\) of the structure is empty. This means that the induced density of the flow is
    \begin{align*}
        q_{\bm z_T} = q_{\bm z_0}(f^{-1}(\bm z_T)) \, \prod_{t=1}^T \lvert a_{t,1} \rvert^{-1}.
    \end{align*}
    This means for instance that the number of modes that we originally have in \(q_{\bm z_0}\) is preserved, as we 
    are multiplying by the same constant for all \(f^{-1}(\bm z_{T})\). This means we need to know a priori the 
    number of modes the target distribution have so we can do the same for our base distribution. However, this breaks
    with the assumptions behind the base distribution. Hence, the flow \(f\) cannot be an UDA.
\end{proof}

It still remains an open question whether or not the affine flow is an UDA for \(D > 1\). 

\subsection{Analytical Inverse Flows are UDA}
Affine flow with FILLER flow-structure is just one of a whole class of non-linear analytically invertible normalizing flows. 
This class can be written as 
\begin{align}\label{eq:aif_flow}
    f_{t, d}(\bm z_{t-1}) = c_{t,d} \cdot h_{t,d}[a_{t,d} \cdot z_{t,d} 
    + b_{t,d}] + d_{t,d},
\end{align}
where \(h_{t,d}\) are piecewise \(C^1\)-diffeomorphisms, analytically invertible, and 
\begin{align*}
    (a_{t,d}, b_{t,d}, c_{t,d}, d_{t,d}) = \mathcal{H}(\mathcal{S}_{ext}(z_{t,d})).
\end{align*}
It is easy to see from this definition
that affine flow is just one of many possible flows that follows \cref{eq:aif_flow}. We shall refer to the flows
that uses transformations on the form \cref{eq:aif_flow} as AIFs.\todo{Write about benefits of introducing h}\\

We now turn our attention to one certain AIF, which we will then show is an UDA. As far as we know, this have never been done. That is,
to show that analytical inverse flows can be UDAs. The proof will be given in full detail, yet we will follow a lot of the same path as
how NAF-DSF was proved to be UDA.
\begin{definition}
    Let \((\mathcal{A}, f)\) be a normalizing flow with corresponding flow-structure \(\mathcal{S}\). Let \(q_{\bm z_0}\) be
    the density that correpsponds to the base distribution \(\mathcal{A}\). A piecewise affine flow (PAF), is defined as
    \begin{align}
        \bm z_0 &\sim q_{\bm z_0} \\
        f_{0,d}(z_{0,d}) &= \sigma(z_{0,d})\\
        f_{t,d}(\bm z_{t-1,d}) &= h_{t,d}(z_{t-1,d} - b_{t,d}) + b_{t,d} \label{eq:h_trans}\\
        f_{T+1,d}(z_{T,d}) &= \hat{\sigma}(z_{T-1,d})\\
        f_{T+2,d}(z_{T+1,d}) &= \sigma^{-1}(z_{T,d}).
    \end{align}
    where
    \begin{align}
        h_{t,d}(x) = 
        \begin{cases}
            a_{t,d}\cdot x, & \text{if \(x > 0\)} \\
            x,& \text{else},
        \end{cases}
    \end{align}
    and 
    \begin{align}
        \hat{\sigma}(x) = 2\left(\sigma(x) - \frac{1}{2}\right).
    \end{align}
    The parameters to estimate in the flow is simply \((a_{t,d}, b_{t,d}) = \mathcal{H}(\mathcal{S}_{ext}(z_{t,d}))\),
    where \(a_{t,d} > 0\), for \(t\in \mathcal{T}\) and \(d \in \mathcal{D}\).
\end{definition}

\todo{Chance flow-structure FILLER word when it is defined prop}
The indexing of steps \(t\) is done such
that \(T\) decides how many transformations of the form \cref{eq:h_trans}. The three other transformations are always there,
but does not depend on any other variables than the one it transforms, and they are a little bit different then. The main reason though for this
particular indexing is for readability during proofs, as the one parameter we can scale is the number of transformations \cref{eq:h_trans}. 

We now give an explanation of the flow. It first samples from our base distribution, then uses the standard logistic function 
\(\frac{1}{1+\exp(-x)}\) to map our samples to \((0,1)\). Note that the transformation can be any as long as
the output is positive (we chose somewhat arbitraily the sigmoid function). 
We then apply \(T\) transformations of \cref{eq:h_trans}. The transformation can be seen as a reverse
leaky ReLU, but where the point that used to be simply \(0\), is decided by \(b_{t,d}\). 
We then, as \(z_{T,d} > 0\) for all 
\(d \in \mathcal{D}\), map the samples \(z_{T,d}\) from \((0,\infty)\) to \((0,1)\) through \(\hat{\sigma}\), before applying 
the standard logit function, i.e inverse to the standard logistic function. 

\begin{proposition}
    Let \((\mathcal{A}, f)\) be a normalizing flow with flow-structure \(\mathcal{S}\). If the flow is an PAF and the structure is 
    a FILLER, then the resulting NF is an analytical inverse normalizing flow.
\end{proposition}
\begin{proof}
    All transformations apart from \cref{eq:h_trans} is well known diffeomorphisms and analytical invertible, no matter
    the flow-structure as the \(\mathcal{S}_{ext}\) is empty for all dimensions given \(t \in \{0,T+1,T+2\}\). Hence, we only need
    to show that the transformation defined in \cref{eq:h_trans} as piecewise \(C^1\)-diffeomorphisms and that the inverse can
    be written in close form. Let \(t \in \{2,\dots,T-1\}\), 
    \(f_{t,d}(z_{t-1,d}) = h_{t,d}(z_{t-1,d} - b_{t,d}) + b_{t,d}\) and assume we know \(z_{t,d}\) and all 
    variables corresponding to \(\mathcal{S}_{ext}(z_{t,d})\). As the structure is a FILLER, means we can guarantee this as long as
    the transformation itself is invertible. We now
    need to show, given this, that we can obtain \(z_{t-1,d}\), and as a closed form expression. As
    \begin{align*}
        \lim_{z_{t,d} \rightarrow b_{t,d}^{-}} f_{t,d} = \lim_{z_{t,d} \rightarrow b_{t,d}^{+}} f_{t,d},
    \end{align*}
    means it is continuous. As the derivative when constrained to \(z_{t-1,d} < b_{t,d}\) is equal to 1, \(z_{t-1,d} > b_{t,d}\) 
    is equal to \(a_{t,d}\) and \(a_{t,d} > 0\) means it is a piecewise \(C^1\)-diffeomorphism.

    The inverse can easily be written in closed form due to the fact that \(z_{t-1,d} > b_{t,d} \iff z_{t,d} > b_{t,d}\). 
    As we can obtain \((a_{t,d}, b_{t,d})\) without knowing \(z_{t-1,d}\), due to the structure being FILLER. Combining these two
    means that, given \(z_{t,d}\), one can obtain \(b_{t,d}\), which gives us \(z_{t,d}-b_{t,d}\). Being larger than 0 or not leads
    us to either divide or not by \(a_{t,d}\), which we have obtained without \(z_{t-1,d}\). Hence, the inverse can be written as
    \begin{align*}
        f_{t,d}^{-1}(z_{t,d}) =
        \begin{cases}
            \frac{z_{t,d} - b_{t,d}}{a_{t,d}} + b_{t,d}, & \text{if \(z_{t,d} - b_{t,d} > 0\)}\\
            z_{t,d}, & \text{otherwise}.
        \end{cases}
    \end{align*}
    Which shows that the transformation is analytically invertible. Hence, the NF with flow being PAF and a structure that is FILLER,
    is a flow in AIF.
\end{proof}

\subsection{Proof of UDA}
We are now going to prove that a normalizing flow with PAF flow and FILLER flow-structure is in fact UDA. The flow-structure we are 
limiting ourselves to is an IAR-structure without permutations, and later on prove it for a larger class of structures. 
As we are interested in the
expressiveness when the number of transformations increase, it makes sense to introduce \(g_{T}\). This refers to the PAF flow without
the last logit transformation. Also, the \(T\) denotes here the number of transformations of \cref{eq:h_trans}, and not the \(T\)th transformation, 
as we are used to with \(f\).  
We also refer to \(\mathcal{H}\) as the function that output the parameters in \(g_{T}\). We are also limiting \(\mathcal{A}\), in this section,
to have density with the property
\begin{align*}
    \begin{cases}
        q_{\bm z_0}(\bm z_0) > 0, & \text{if \(\bm z_0 \in [k_0,k_1]^D\)}\\
        q_{\bm z_0}(\bm z_0) = 0, & \text{otherwise},
    \end{cases}
\end{align*}
where \(k_0 < k_1\) and \(k_0, k_1 \in \R\).

We start by proving that \(g_T\) can, for one dimension, find a flow that converges to any monotonically increasing function from a 
compact set \([k_0, k_1]\) to \([l_0,l_1]\), where the endpoints \(k_0,k_1\) maps to \(l_0,l_1\) respectively and the functions image
is between 0 and 1 including. 
This is the first step to show UDA, which will rely on approximating the conditional cumulative
distribution, before applying the logit function in the last step of the PAF flow.

\begin{lemma}\label{lemma:paf_one_dim}
    Let \((\mathcal{A}, g_T)\) be a PAF without the \(T+2\) transformation, with an IAR-structure and dimension \(D=1\).
    Let \(g\colon [k_0,k_1] \rightarrow [l_0,l_1]\) be a 
    monotonically increasing function with \(g(k_0) = l_0\) and \(g(k_1) = l_1\), and \(0 \leq l_0 < l_1 \leq 1\) . 
    Then there exist a flow on the form \(g_T\) that converges uniformly to \(g\), when \(T \rightarrow \infty\). The flow can also be approximated 
    arbitrarily well by a neural network with \(4\) neurons in each hidden layer.
\end{lemma}
\begin{proof}
    For any \(\epsilon > 0\), we set \(M = \ceil{\frac{1}{\epsilon}}\). Start by dividing \([0,1]\) into \(M+1\)
    subsets 
    \begin{align*}
        \left(l_0,l_0 + \frac{l_1-l_0}{M+1}\right), \left(l_0+\frac{l_1-l_0}{M+1}, l_0 + \frac{2(l_1-l_0)}{M+1}\right),\dots, 
        \left(l_0 +\frac{M(l_1-l_0)}{M+1}, l_1\right). 
    \end{align*}
    We shall denote the boundary points \(l_0 + \frac{m(l-1-l_0)}{M+1} = y_{m}\) for \(m\in \{1,2,\dots,M\}\). 
    We can also find \(x_m = g^{-1}(y_m)\), where 
    \begin{align*}
        g^{-1}(y_m) = \inf\{x_m \mid g(x_m) = y_m,\, \forall x_m \in [k_0,k_1]\}.
    \end{align*}
    We then let number of transformations of \cref{eq:h_trans} be \(T = M + 2\).
    The goal now to show that there exist a mapping \((a_t, b_t)^{T}_{t=1} = 
    \mathcal{H}(\mathcal{S}_{ext})\), such that \(\lvert g_T(z_0) - g(z_0)\rvert < \epsilon\) for all \(z_0 \in [k_0,k_1]\). 

    Let \(b_1 = 0\) and
    \begin{align*}
        a_1 =
        \begin{dcases}
            \frac{\hat{\sigma}^{-1}(l_0)}{\sigma(k_0)} & \text{ if \(\sigma(k_0) \geq \hat{\sigma}^{-1}(l_0)\) }\\
            1 & \text{ otherwise. }
        \end{dcases}
    \end{align*}
    This ensures that for all \(z_0 \in [k_0, k_1]\) we have \(g_T(z_0) \geq l_0\).
    We then uses the next \(M\) transformations to create a flow that maps \(g_T(x_m)=y_m\). This is done by handling one
    and one \(x_m\), while setting the \(b_t\) in a manner that ensures we do not change the previous \(x_{1:m-1}\), as well as
    not change \(g_T(k_0) \geq l_0\). Quick reminder that applying the flow for \(t \in \mathcal{T}\) will first apply the standard sigmoid function to 
    input, as \(f_0(z_0) = \sigma(z_0)\). We can then define the parameters \((a_t,b_t)^{M+1}_2\) iteratively, for \(t \in \{2,\dots,M+1\}\) and 
    letting \(m=t-1\), as 
    \begin{align*}
        b_t =
        \begin{cases}
            \hat{\sigma}^{-1}(y_{m-1}) & \text{ if \(t > 2\) } \\
            \hat{\sigma}^{-1}(l_0) & \text{ otherwise }
        \end{cases}
    \end{align*}
    and
    \begin{align*}
        a_t = \frac{\hat{\sigma}^{-1}(y_m) - b_t}{\left(\bigcirc_{j=0}^{t-1} f_j(x_m)\right) - b_t}.
    \end{align*}
    The transformations from \(j=0\) to \(t-1\) are parameterized with the parameters already found, hence the iterative definition.
    Also, notice that the parameters does not use the input of \(g_T\), they are constant. That is, after we have set a target \(g\) and 
    number of transformations \(T\), we have a mapping which is constant w.r.t input. To set it in a NF perspective, 
    the choice of \(T\) is always set before approximating \(\mathcal{H}\) and target \(g\) is induced by the target distribution, which
    is also "set" before approximating \(\mathcal{H}\), hence it is not dependent on the input to the flow, \(z_0\). 

    After \(M+1\) transformations using \cref{eq:h_trans} with the parameters described above, we simply need on last transformation. By
    setting 
    \begin{align*}
        b_T = \hat{\sigma}^{-1}(y_M), 
    \end{align*}
    and 
    \begin{align*}
        a_T = 
        \begin{dcases}
            \frac{\hat{\sigma}^{-1}(l_1) - b_T}{\left(\bigcirc_{j=0}^{T-1}f_j(k_1)\right) - b_T}, & \text{ if \(\hat{\sigma}^{-1}(l_1) \leq \bigcirc_{j=0}^T\, 
            f_j(k_1)\) }\\
            1, & \text{ otherwise. }
        \end{dcases}
    \end{align*}
    Here we use the same logic as with the first transformations, but now ensuring that \(g_T(z_0) \leq l_1\).
    We now have the final transformation with the property that for all \(x_m\) with \(m \in \{1,\dots, M\}\), we have
    \begin{align*}
        g_T(x_t) = y_t.
    \end{align*}
    To show convergence of \(g_T\) and \(g\), we simply see that \(g_T(x_m) - g_T(x_{m-1}) = \frac{l_1 - l_0}{M+1}\) for all \(m \in \{1,\dots,M\}\) 
    and
    \(g_T(x_1)-l_0 = l_1 - g_T(x_M) = \frac{l_1-l_0}{M+1}\). We also know that \(l_0\leq g_T(z_0) \leq l_1\), for all \(z_0 \in [k_0,k_1]\).
    Using the fact that both functions \(g_T\) and \(g\) are monotonically increasing 
    and \(l_1 -l_0 \leq 1\), means that for all \(z_0 \in [k_0, k_1]\),
    \begin{align*}
        \lvert g_T(z_0) - g(z_0) \rvert \leq \frac{l_1-l_0}{M+1} < \frac{l_1-l_0}{M} \leq \frac{1}{M}
        = \frac{1}{\ceil{\frac{1}{\epsilon}}} 
        \leq \epsilon
    \end{align*}
    As \(D=1\) means that the mapping must be constant for all \(x\in [k_0, k_1]\), as \(\mathcal{S}_{ext}(t,1) = \emptyset\). This also 
    means that the function \(\mathcal{H}\) is continuous with input from a compact set, hence we can approximate it arbitraily well
    with a neural network following \cref{thm:uni_nn}.
\end{proof}
We now wish to extend the results above for higher dimensions, while also including the IAR-structure. We define a new 
function for each \(d \in \mathcal{D}\), \(G_d(z_{0,d}, \bm z_{0,1:d-1})\), where \(\bm z_0 \in [k_0,k_1]^D\). 
When \(\bm z_{1:d-1}\) is fixed, the function \(G_d\) is a \emph{strictly} monotonically increasing function w.r.t \(z_{0,d}\), 
where \(l_0^{d} \leq G_d(z_{0,d}, \bm z_{0,1:d-1}) \leq l^d_1\), with \(0 \leq l_0^d < l_1^d \leq 1\).
It is also continuous
w.r.t \(\bm z_{0,1:d-1}\), i.e given \(\bm z_{0,1:d-1}\), for all \(\epsilon > 0\) there exist a \(\delta > 0\) such that
\begin{align}\label{eq:continG}
    &\lvert\lvert \bm z_{0,1:d-1} - \tilde {\bm z}_{0,1:d-1} \rvert\rvert_{\infty} < \delta\\ &\implies 
    \lvert\lvert G_d(z_{0,d}, \bm z_{0,1:d-1}) - G_d(z_{0,d} ,\tilde{\bm z}_{0,1:d-1})\rvert\rvert_{\infty} < \epsilon,\nonumber
\end{align}
where \(\tilde z_{0,1:d-1} \in [k_0,k_1]^{d-1}\). In addition to this, we allow for the end points \(l_0^d, l_1^d\) to change when \(z_{0,1:d-1}\) changes.
However, they must be between 0 and 1 incuding, and \(l_0^d < l_1^d\) and due to the continuity, the changes must be continuous as well. 
To be more precise, there must be continuity for the point \(k_0\), i.e 
\(G_d(k_0, z_{0,1:d-1}) = l^d_0\), and equivalently for \(k_1, l_1^d\). To quickly summarize the properties of \(G_d\):
%\(G_d(z_{0,d},\bm z_{0,1:d-1})\) is 
\begin{itemize}
    \item For each \(\bm z_{0,1:d-1}\): 
        \begin{itemize}
            \item \(\exists\, l_0^d, l_1^d \in [0,1]\) such that \(l_0^d < l_1^d\) and \(G_d\colon [k_0,k_1] \rightarrow [l_0^d, l_1^d]\).
            \item \(G_d\) is \emph{strictly} monotonically increasing.
            \item \(G_d(k_0, \bm z_{0,1:d-1}) = l_0^d\) and \(G_d(k_1,\bm z_{0,1:d-1}) = l_1^d\)
        \end{itemize}
    \item \(G_d\) is continuous w.r.t \(\bm z_{0,1:d-1}\), fulfilling \cref{eq:continG}.
    \item The boundary points in the image of \(G_d\) may change when \(\bm z_{1:d-1}\) changes, 
        but according to the points above, must do so continuously, with regards
        to the same \((\delta,\epsilon)\) as in \cref{eq:continG}.
\end{itemize}
We wish to show we can converge towards \(G = (G_1, G_2, \dots, G_D)\) using PAF, similarly as in \cref{lemma:paf_one_dim}, by
using the aformentioned lemma. However, we first need to make sure \((a_t,b_t)_{t=1}^T = \mathcal{H}_d(\bm z_{1:d-1})\), specified in the proof of
\cref{lemma:paf_one_dim}, are continuous w.r.t \(\bm z_{1:d-1}\). This to confirm we can approximate \(\mathcal{H}_d\) arbitraily well using a neural network.
\begin{lemma}\label{lemma:contin_param_paf}
    Let \((\mathcal{A}, g_T)\) be a PAF without the \(T+2\) tranformation, with an IAR-structure. Let the function \(G\) be defined as the function
    where each of the \(D\) outputs is defined by \(G_d\), i.e \(G = (G_1, G_2, \dots, G_D)\). Then there exist a flow on the form \(g_T\), with
    continuity in \(\mathcal{H}_d(\bm z_{0,1:d-1})\) for all \(d \in \mathcal{D}\), which 
    converges uniformly to \(G\). 
\end{lemma}
\begin{proof}
    For all \(\bm z_{0,1:d-1}\) and for all \(\epsilon > 0\), setting \(M = \frac{1}{\ceil{\epsilon}}\) and 
    letting \(\mathcal{H}_d\) output the parameters specified in the proof of 
    \cref{lemma:paf_one_dim},  gives uniform convergency to \(G_d\) due to \cref{lemma:contin_param_paf}.
    Using the same \(M\) for all \(d \in \mathcal{D}\) and designing \(\mathcal{H}_d\) as mentioned, gives uniform convergence for all \(G_d\)'s, hence
    there exist a flow \((\mathcal{A}, g_T)\) which converges uniformly towards \(G\). We therefore only need to show continuity in \(\mathcal{H}_d\) as 
    described, for all \(d \in \mathcal{D}\).

    We show continuity for an arbitrary \(d \in \mathcal{D}\setminus \{1\}\), see \cref{lemma:paf_one_dim} for continuity when \(d=1\). 
    Let \(\epsilon_1 > 0\). For all \(\epsilon_2 > 0\) there exist a \(\delta_2 > 0\) such that that whenever
    \begin{align*}
        \lvert\lvert \bm z_{0,1:d-1} - \tilde{\bm z}_{0,1:d-1}\rvert\rvert_{\infty} < \delta_2
    \end{align*}
    implies \(\lvert l_0^d - \tilde{l}_0^d\rvert < \epsilon_2\) and \(\lvert l_1^d - \tilde{l}_1^d\rvert < \epsilon_2\), 
    due to continuity in \(G_d\). This also implies \(\lvert y_m - \tilde y_m \rvert < 2 \,\epsilon_2\) and combined with the fact that
    \(\hat{\sigma}^{-1}\) is continuous, means we can choose \(\epsilon_2\) such that 
    \(\lvert \hat{\sigma}^{-1}(y_m) - \hat{\sigma}^{-1}(\tilde y_m)\rvert < \epsilon_1\) for all \(m \in \{1,2,\dots,M\}\). 
    Setting then \(\delta_1 = \delta_2\) gives us
    \begin{align*}
        \lvert\lvert \bm z_{0,1:d-1} - \tilde{\bm z}_{0,1:d-1}\rvert\rvert_{\infty} < \delta_1
        \implies \lvert\lvert \bm b - \tilde{\bm b}\rvert\rvert_{\infty} < \epsilon_1,
    \end{align*}
    where \(\bm b\) and \(\tilde{\bm b}\) are vectors with the correpsonding \((b_t)_{t=1}^T\) for \(G_d(z_{0,d}, \bm z_{0,1:d-1})\) and 
    \(G_d(z_{0,d}, \tilde{\bm z}_{0,1:d-1})\) respectively. We now simply need 
 \end{proof}






