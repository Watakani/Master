\section{Flow Structure}\label{seq:struct}
When constructing transformations \(f\) one has to choose 
the form of the function \(f_{t,d}\), as well as the structure. By
structure we mean what variables \(z_{i,j}\) is needed to calculate the
transformation. This leads to the unfortunate situation that one speaks 
about a particular flow, while intending only to speak of the structure. We
decouple these two, and introduce the form of the function in the next section.
Some of the most popular NF's can use a myriad of structures, which we shall
formalizer here. So while some flows in the literature only allow for a
specific structure, others allow for a larger class of them.

We introduce, for ease of readibility, \(\mathcal{T} = \{1,2,\dots,T\}\) and
\(\mathcal{D} = \{1,2,\dots, D\}\).
\begin{definition}\label{def:struct}
    Let \((\mathcal{A}, f)\) be a normalizing flow.
    A \emph{flow-structure} is defined as a mapping 
   \begin{align*}
       \mathcal{S}\colon \mathcal{T} \times \mathcal{D} \rightarrow 
       \mathcal{P}(\mathcal{T} \times \mathcal{D}),
    \end{align*}
    such that \(\mathcal{S}(t,d)\) is the set indicating 
    which variables in the flow that are used to calculate \(z_{t,d}\).
\end{definition}
\begin{remark}
    Obviously, the set \(\mathcal{S}(t,d)\) for any \((t,d)\) is never 
    empty, as it is always dependent on \((t-1,d)\). There are cases where it is 
    useful to talk about the flow-structure without the element \((t-1,d)\), in
    which we refer to the mapping as \(\mathcal{S}_{ext}\). That is,
    \begin{align*}
        \mathcal{S}_{ext}(t,d) = \mathcal{S}(t,d)\,\textbackslash\, \{(t-1,d)\}.
    \end{align*}
    Elements \((i,j) \in \mathcal{S}_{ext}(t,d)\) is such that \(f_{t,d}(z_{t-1,d})\) also
    uses \(z_{i,j}\). The way it is included in the transformation \(f_{t,d}\)
    can vary a lot, as we shall see in the next section. However, thinking about
    the definition and goals of NF, one quickly finds out that there are limitations
    on \(\mathcal{S}\). Both in terms of invertibility, but also computationally w.r.t
    determinant etc. Before we explore this any further, we introduce a different 
    way to view \(\mathcal{S}\); as a graph.
\end{remark}
Let \((\mathcal{A}, f)\) be a normalizing flow with an accompanied flow-structure
\(\mathcal{S}\). We define a graph \(G\) with verticies \(V = \mathcal{T} \times \mathcal{D}\)
and directed edges \(E\) with an edge \((t,d)\) from all verticies in \(\mathcal{S}(t,d)\). It is
sometimes useful to separate edges from \((t-1,d)\) to \((t,d)\) from the rest, here as well. We shall therefore
refer to edges on the former form to be in \(E_{int}\) and the rest in \(E_{ext}\), with 
\(E = E_{int} \cup E_{ext}\)\footnote{\(int\) is short for interior, and refers to the fact 
that \(z_{t,d}\) is directly transformed from the interior variable. While other
variables gives "help" with the transformation from the "outside", hence exterior (\(ext\)).}. We abuse
the notation somewhat, and allow \(\mathcal{S}\) to both be referred to as the mapping in \cref{def:struct} and
also the graph it induces, with the context deciding which one. Typically, if we speak about \(\mathcal{S}\)
itself, we tend to do it through graph \(G\). When we are talking about a specific variable and what it is 
dependent on, we refer to the map \(\mathcal{S}\) and the set it outputs for a given variable. 
\subsection{DAGs}
\todo{write about why the flow-structure must be a DAG}
\subsection{Triangular}
\todo{define generally triangluar structures and prove triangular jacobian}
Two important examples of triangular flow-structures, of which are often used in the literature
(\cite{maf}, \cite{iaf}, \cite{naf}), are based off autoregressive models. 
\begin{definition}
    Let \((\mathcal{A}, f)\) be a NF with accompanying flow-structure \(\mathcal{S}\). If
    there exist a permutation \(\phi_t\colon \mathcal{D} \rightarrow \mathcal{D}\)
    for all \(t \in \mathcal{T}\) such that
    \begin{align*}
        \mathcal{S}_{ext}(t,\phi_t(d)) = \{(t,i) \mid i \in \mathcal{D} \text{ and } \phi_t(i) < \phi_t(d)\}  
    \end{align*}
    then \(\mathcal{S}\) is an \emph{autoregressive flow-structure} (AR flow-structure).
\end{definition}
The autoregressive flow-structure simply says that the transformation of \(z_{t,d}\) is all based on
all the variables already transformed and \(z_{t-1,d}\), where the ordering of the dimensions is decided
by the permutation \(\phi_t\). If, for all transformations, we have that the \(\phi_t\) is the idendity permutation,
then we call it AR-structure without permutation. Including permutation will in some cases be crucial to 
allow for flexible transformations in all dimensions, and can be seen as a form of information sharing between
the dimensions. We also include the inverse AR-structure, which can be interpreted as AR-structure when we
do inverse flow.
\begin{definition}
    Let \((\mathcal{A}, f)\) be a NF with accompanying flow-structure \(\mathcal{S}\). If
    there exist a permutation \(\phi_t\colon \mathcal{D} \rightarrow \mathcal{D}\)
    for all \(t \in \mathcal{T}\) such that
    \begin{align*}
        \mathcal{S}_{ext}(t,\phi_t(d)) = \{(t-1,i) \mid i \in \mathcal{D} \text{ and } \phi_t(i) < \phi_t(d)\},
    \end{align*}
    then \(\mathcal{S}\) is an \emph{inverse autoregressive flow-structure} (IAR flow-structure).
\end{definition}
